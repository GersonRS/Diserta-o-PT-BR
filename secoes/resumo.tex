\setlength{\absparsep}{18pt} % ajusta o espaçamento dos parágrafos do resumo
\begin{resumo}
 
A astronomia de ondas gravitacionais colocou em movimento uma revolução científica. Para aumentar ainda mais o alcance científico desse campo emergente, há uma necessidade premente de aumentar a profundidade e a velocidade dos algoritmos de ondas gravitacionais que permitiram essas descobertas inovadoras. Técnicas de aprendizado de máquina, como redes neurais artificiais, já estão transformando muitos campos tecnológicos e também provaram ser bem-sucedidas na astrofísica das ondas gravitacionais para detecção e caracterização de sinais de ondas gravitacionais de buracos negros binários e estrelas de nêutrons. Para contribuir com esse esforço, foi criado um novo método altamente escalonável e computacionalmente leve para processamento de sinais de séries temporais end-to-end, baseado em um sistema de comitê de redes neurais, para classificar rapidamente sinais de ondas gravitacionais de fusão de buracos negros em fluxos de dados de séries temporais altamente ruidosos dos dados típicos dos detectores de ondas gravitacionais do LIGO. E espera-se que este método supere significativamente as técnicas convencionais de aprendizado de máquina, obtendo desempenho similar em comparação com a filtragem combinada, sendo várias vezes mais rápido, permitindo o processamento em tempo real de grandes volumes de dados brutos com recursos mínimos. Esses resultados demonstram o potencial de redes neurais artificiais para a detecção em tempo real de sinais de ondas gravitacionais, o que é essencial para um acompanhamento imediato para facilitar as pesquisas em tempo real de fontes de ondas gravitacionais e suas contrapartes eletromagnéticas e astro-partícula que acompanham esses importantes eventos.

 \textbf{Palavras-chave}: Ondas Gravitacionais, Aprendizado de Maquina, Buracos Negros
\end{resumo}

