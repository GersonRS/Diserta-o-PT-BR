\setlength{\absparsep}{18pt} % ajusta o espaçamento dos parágrafos do resumo
\begin{resumo}
 
A astronomia de ondas gravitacionais colocou em movimento uma revolução científica. Para aumentar ainda mais o alcance científico desse campo emergente, há uma necessidade premente de aumentar a profundidade e a velocidade dos algoritmos de ondas gravitacionais que permitiram essas descobertas inovadoras. Para contribuir com esse esforço, pretende-se criar  um novo método altamente escalonável para processamento de sinais de séries temporais end-to-end, baseado em um sistema de redes neurais, para classificação de sinais em fluxos de dados de séries temporais altamente ruidosos. Demonstrando um novo esquema de treinamento com níveis de ruído gradualmente crescentes. Validar a aplicação deste método para a detecção de ondas gravitacionais a partir de fusões binárias de buracos negros. E espera-se que este método supere significativamente as técnicas convencionais de aprendizado de máquina, obtendo desempenho similar em comparação com a filtragem combinada, sendo várias vezes mais rápido, permitindo o processamento em tempo real de grandes volumes de dados brutos com recursos mínimos. Mais importante, o método proposto amplia a gama de sinais de ondas gravitacionais que podem ser detectados com detectores de ondas gravitacionais terrestres. Essa estrutura aproveita os avanços recentes em algoritmos de inteligência artificial e em arquiteturas de hardware emergentes, como GPUs otimizadas para aprendizado de maquina, para facilitar as pesquisas em tempo real de fontes de ondas gravitacionais e suas contrapartes eletromagnéticas e astro-partícula.

 \textbf{Palavras-chave}: Ondas Gravitacionais, Aprendizado de Maquina, Learning Machine, Buracos Negros, Ruídos.
\end{resumo}

