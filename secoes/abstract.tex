\begin{resumo}[Abstract]
 \begin{otherlanguage*}{english}

Gravitational-wave astronomy has set in motion a scientific revolution. To further extend the scientific reach of this emerging field, there is a pressing need to increase the depth and speed of the gravitational wave algorithms that have enabled these groundbreaking discoveries. Machine learning techniques, such as artificial neural networks, are already transforming many technological fields and have also proven successful in gravitational wave astrophysics for detecting and characterizing gravitational wave signals from binary black holes and neutron stars. To contribute to this effort, a new highly scalable and computationally lightweight end-to-end time-series signal processing method based on a neural network committee system has been created to rapidly classify hole-fusion gravitational wave signals. noisy time-series data streams from the typical LIGO gravitational wave detector data. And it is expected that this method significantly outperforms conventional machine learning techniques, achieving similar performance compared to combined filtering, and is several times faster, allowing real-time processing of large volumes of raw data with minimal resources. These results demonstrate the potential of artificial neural networks for real-time detection of gravitational wave signals, which is essential for immediate follow-up to facilitate real-time research of gravitational wave sources and their electromagnetic and particle counterparts what accompany these important events.

\textbf{Keywords}:  Gravitational Waves, Machine Learning, Black Holes.
 \end{otherlanguage*}
\end{resumo}