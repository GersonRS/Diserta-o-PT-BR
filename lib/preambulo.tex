\usepackage{graphicx}                               % Inserir figuras
\usepackage{amsfonts, amssymb, amsmath, amsthm, exscale, ifthen}             % Fonte e símbolos matemáticos
\usepackage{booktabs}                               % Comandos para tabelas
\usepackage{verbatim}                               % Texto é interpretado como escrito no documento
\usepackage{multirow, array}                        % Múltiplas linhas e colunas em tabelas
\usepackage{indentfirst}                            % Endenta o primeiro parágrafo de cada seção.
\usepackage{listings}                               % Utilizar codigo fonte no documento
\usepackage{xcolor}
\usepackage{microtype}                              % Para melhorias de justificação?
\usepackage[portuguese,ruled,lined]{algorithm2e}    % Escrever algoritmos
\usepackage{algorithmic}                            % Criar Algoritmos  
\usepackage{float}                                  % Utilizado para criação de floats
\usepackage{amsgen}
\usepackage{lipsum}                                 % Usar a simulação de texto Lorem Ipsum
\usepackage{tocloft}                                % Permite alterar a formatação do Sumário
\usepackage{etoolbox}                               % Usado para alterar a fonte da Section no Sumário
\usepackage[nogroupskip,nonumberlist,acronym]{glossaries}                % Permite fazer o glossario
\usepackage{caption}                                % Altera o comportamento da tag caption
\usepackage[abnt-repeated-author-omit=yes,alf,abnt-emphasize=bf,bibjustif,recuo=0cm,abnt-etal-cite=2,abnt-etal-list=0]{abntex2cite}  % Citações padrão ABNT
%\usepackage[bottom]{footmisc}                      % Mantém as notas de rodapé sempre na mesma posição
\usepackage[bottom=2cm,top=3cm,left=3cm,right=2cm]{geometry}
%\usepackage{lmodern}                               % Usa a fonte Latin Modern
%\usepackage{subfig}                                % Posicionamento de figuras
%\usepackage{scalefnt}                              % Permite redimensionar tamanho da fonte
\usepackage{color, colortbl}                       	% Comandos de cores
%\usepackage{lscape}                                % Permite páginas em modo "paisagem"
%\usepackage{ae, aecompl}                           % Fontes de alta qualidade
%\usepackage{picinpar}                              % Dispor imagens em parágrafos
\usepackage{latexsym}                              	% Símbolos matemáticos
%\usepackage{upgreek}                               % Fonte letras gregas
\usepackage{appendix}                               % Gerar o apendice no final do documento
\usepackage{paracol}                                % Criar paragrafos sem identacao
\usepackage{lib/uecetex2}		                    % Biblioteca com as normas da UECE para trabalhos academicos
\usepackage{pdfpages}                               % Incluir pdf no documento
\usepackage{amsmath}                                % Usar equacoes matematicas
\usepackage{footnote}                               % Permite footnote em tabelas
\usepackage{subfigure}

\usepackage{longtable}

\setlength\extrarowheight{10pt}                     % Definir altura de linhas das tabelas.

% Organiza e gera a lista de abreviaturas, simbolos e glossario
\makeglossaries

% Gera o Indice do documento
\makeindex

\hyphenation{bacharelado universidade computação garanhuns}

\newcommand{\ok}{$\surd$}
\newcommand{\cf}{\emph{Cloud Foundry}~}

% Definição
\newtheorem{defi}{Definição}

\usepackage{calligra}

% Hifenização
\usepackage{hyphenat}
\tolerance=1
\emergencystretch=\maxdimen
\hyphenpenalty=10000
\hbadness=10000
\hyphenchar\font=-1
\sloppy
