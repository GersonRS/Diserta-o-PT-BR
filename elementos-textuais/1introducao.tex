\chapter{Introdução}
\label{cap:introducao} 

As Ondas Gravitacionais (\textit{Gravitational waves}) (GW) foram uma previsão da teoria da relatividade geral de Einstein em 1915 \cite{albert1920realtivity}. Devido a seus efeitos mensuráveis serem minúsculos, os cientistas levaram vários anos para detectar suas consequências mais convincentes \cite{cervantes2016brief}. Em 2015, essas pequenas ondulações no espaço-tempo foram detectadas pelo avançado Observatório de Ondas Gravitacionais por Interferometria a Laser (\textit{Advanced Laser Interferometer Gravitational Wave Observatory}) (aLIGO) \cite{PhysRevLett.116.131103, 0264-9381-32-7-074001}.

Dez observações de fusão de buracos negros foram feitas até o final de 2019 \cite{Abbott_2019}, as quais permitiram estudos das propriedades populacionais dos buracos negros de massa estelar e permitiram a realização de testes de precisão da relatividade geral \cite{collaboration2018binary}. E esse grande avanço científico, digno do Prêmio Nobel de Física de 2017, deu inicio a uma nova era na astronomia e astrofísica \cite{huerta2017boss}.

Há um investimento muito forte nesta nova areá emergente de pesquisa, com previsão de que, nos próximos anos novos observatórios no Japão e na Índia entrem em operação, formando uma grande rede de detectores em evolução capaz de observar centenas de fontes de ondas gravitacionais todos os anos \cite{PhysRevD.100.063015}.

Concomitantemente, o aparecimento de novas tecnologias na área da computação provocou um aumento significativo nas pesquisas, em especial, de novos algoritmos computacionais em Aprendizado de Máquina (\textit{Machine Learning}), que é uma técnica de Inteligência Artificial (IA) que se vale de um método de análise de dados a partir da construção e automatização de modelos analíticos \cite{bisong2019building}. Esta tecnologia vem sendo frequentemente usada para auxiliar em vários procedimentos nas mais diversas áreas como, por exemplo, a utilização de IA na realização do processamento de imagens de lesões dermatológicas ou exames radiológicos, de ultrassom, de ressonância magnética, de tomografia, dentre outros, a fim de realizar a automação de diagnósticos patológicos, facilitando assim o trabalho dos profissionais, permitindo que os laboratórios mantenham a qualidade ao manipular o elevado número de amostras diárias \cite{lobo2017inteligencia}.

Em destaque a estas pesquisas podem-se citar as Redes Neurais Artificiais (RNA). As RNAs são técnicas computacionais que visam imitar o funcionamento dos neurônios de organismos inteligentes através de modelos matemáticos. Com esses modelos, é possível treinar, testar e validar a RNA e utilizar a rede treinada para simulações, reconhecimento de padrões, previsão, entre outras aplicações \cite{haykin2011neural}.

Sendo uma dessas aplicações a utilização de Redes Neurais Profundas (\textit{Deep Learning}), uma derivação das redes neurais, na realização do processamento de ondas gravitacionais, detectadas pelo aLIGO \cite{shen2017denoising,PhysRevLett.120.141103,krastev2019real,gebhard2019convolutional,mukund2017transient,kim2015application, george2017deep, george2018deep, PhysRevD.100.103025, Lin2019, Luo2020}, a fim de agilizar as observações desses eventos e rapidamente disseminar suas informações aos laboratórios parceiros para maximizar a chance de observações eletromagnéticas.

Apesar desses resultados preliminares promissores no processamento de ondas gravitacionais utilizando \textit{Deep Learning}, \cite{PhysRevD.100.063015} acreditam que o papel preciso que a IA possa desempenhar dentro do escopo mais amplo das pesquisas de coalescências binárias compactas (CBCs) e da astronomia prática de ondas gravitacionais ainda não foi testado em detalhes suficientes.

No entanto, isto apenas motiva a analise cuidadosa sobre o potencial prático do uso da IA para a procura de GWs de CBCs. Ocasionando, uma crescente busca de soluções cada vez mais eficientes que tenha uma identificação confiável e precisa em tempo real, como é o exemplo de vários trabalhos desenvolvidos nos últimos anos com este objetivo.
    
\section{Delimitação do Tema}
\label{sec:delimitacao-do-tema}

A natureza nem sempre pode ser observada em sua plenitude, mas sempre poderá ser simulada computacionalmente desde que existam os apropriados modelos computacionais. Neste sentido, a computação científica e numérica desempenha o papel de ferramenta fundamental da área da ciência da computação para o entendimento da natureza, explicitando fenômenos e dinâmicas muitas vezes não possíveis de serem observados pela limitada capacidade humana de observação.

Capacidade essa que também não pode acompanhar a avalanche de informações produzidas por muitos dos experimentos de astrofísica e astronomia de hoje. Os quais, alguns deles registram terabytes de dados todos os dias e a quantidade de dados produzida a cada dia está apenas aumentando. O \textit{Square Kilometer Array}, um radiotelescópio programado para realizar as primeiras observações científicas em meados de 2020, será capaz de gerar um tráfego de dados maior que a Internet inteira a cada ano \cite{ska-dewdney2009square, scaife2020big, greig2020reionization}.

Por esse motivo, os pesquisadores passaram a depender de tecnologias integradas para serem capazes de lidar com esse grande fluxo de dados e o uso efetivo desses dados tornou possível aprimorar os métodos de pesquisa, transformando a forma de se fazer ciência. E a RNA se insere nesse contexto justamente para auxiliar os pesquisadores nessa missão. Ela nos permite extrair novos \textit{insights} das informações disponíveis para melhorar a busca de anomalias e detectar padrões que humanos nunca poderiam ter visto.

Existe uma grande gama de problemas que podem ser estudados e resolvidos com as RNAs, porém não existe um modelo pronto de rede que se encaixe na resolução de todos os problemas, cada RNA é construída com base no problema que se deseja resolver, o que em muitos casos torna a construção da rede neural um processo complexo que requer tempo e experiência para a construção de uma rede que consiga atingir bons resultados \cite{barros2018avaliaccao}. 

Ao construir uma RNA é necessário configurar os muitos parâmetros que a compõem, que podem variar de problema para problema e influenciar o resultado positivamente ou negativamente, tais como: dados de entrada, quantidade de camadas ocultas, a quantidade de neurônios em cada camada oculta, o bias, funções de ativação, realimentação da rede, taxa de aprendizagem, entre outros parâmetros dependentes do tipo treinamento da RNA. Existem diversos algoritmos de treinamento, cada um com seus prós e contras, que se encaixam nas diversas categorias de problemas, utilizar o algoritmo correto para a categoria de problemas certa, faz com que a rede consiga solucionar o problema desejado e seja confiável.

\section{Problema de Pesquisa}
\label{sec:problema-de-pesquisa}

A busca pela detecção de ondas gravitacionais continua intensa, a análise e o tratamento computacional destas informações encontram-se na crisa da onda da tecnologia computacional para o entendimento de tal processo.

Para que seja possível a detecção de ondas gravitacionais é necessário um grande esforço contínuo dos detectores e das instalações astronômicas, a natureza sensível a tempo dessas análises requer algoritmos que possam detectar e caracterizar esses eventos em tempo real. A técnica de filtragem combinada(\textit{Matched filtering}, na sigla em inglês), que é uma técnica de processamento de sinais, em que, correlaciona um sinal atrasado conhecido, ou forma de onda, com um sinal desconhecido para detectar a presença da forma de onda no sinal desconhecido \cite{turin1960introduction, Schutz_1999}, provou ser bem-sucedida na descoberta de GWs de CBC a partir dos dados dos observatórios de ondas gravitacionais aLIGO e Advanced Virgo (aVIRGO) \cite{gebhard2019convolutional}.

No entanto, o custo computacional das buscas por GWs usando as atuais técnicas de filtragem combinadas, aumenta significativamente quando é necessário a busca por sinais em uma grande quantidade de dados \cite{abbott2018prospects}.

Por outro lado, por mais que os métodos que utilizam \textit{Deep Learning} mostrem que podem realmente ser efetivamente aplicados a esse problema ao tratá-lo como uma tarefa de classificação binária \cite{gebhard2019convolutional}, ainda necessitam do uso da computação de alto desempenho (HPC do inglês, \textit{high performance computing}), limitando as pesquisas a grupos e laboratórios com acesso a super computadores. Além disso, a própria natureza das redes neurais profundas a tornam um processo custoso quanto ao seu preparo e desenvolvimento. No caso das redes neurais profundas o treinamento pode demorar muito tempo e demandar grandes recursos computacionais.

Para aprimorar ainda mais o alcance científico desse campo emergente, há uma necessidade premente de aumentar a confiabilidade e a velocidade dos algoritmos de ondas gravitacionais que permitiram essas descobertas inovadoras.

\section{Objetivos}
\label{sec:objetivos}

A seguir serão apresentados os objetivos geral e específicos que nortearão a condução desta pesquisa.

\subsection{Objetivo Geral}
\label{sec:objetivo-geral}

A presente pesquisa tem como objetivo geral o desenvolvimento de um classificador binário simples capaz de classificar de forma eficiente as reais medidas das ondas gravitacionais disponíveis pelo aLIGO, baseado em um Comitê de Redes Neurais Artificiais.

\subsection{Objetivo Específico}
\label{sec:objetivo-especifico}

Para atingir o objetivo geral foram definidos os seguintes objetivos específicos: 
\begin{itemize}

\item Investigar e obter os modelos das ondas gravitacionais descobertas pelo aLIGO, disponíveis no GWOSC \cite{vallisneri2015ligo} e simular computacionalmente as ondas gravitacionais para a criação do conjunto total de dados;
\item Desenvolver arquiteturas de redes neurais que abranja o problema proposto.
\item Ajustar os parâmetros que melhor se ajustem para o treinamento e validação da RNA;
\item Desenvolver um Comitê de Redes Neurais Artificiais com as arquiteturas propostas para a classificação de sinais de ondas gravitacionais;
\item Validar o Comitê de Redes Neurais Artificiais com as reais medidas das ondas gravitacionais do aLIGO.

\end{itemize}

\section{Justificativa}
\label{sec:justificativa}


Com base em todas as considerações descritas, precisamos de um novo paradigma para superar as limitações e os desafios computacionais dos algoritmos de detecção de GW existentes, eliminando assim a necessidade do uso de super computadores, e libertando a pesquisa de ondas gravitacionais a qualquer ambiente computacional.

Um candidato ideal para este feito seria a \textit{Machine Learning}, mais especificamente um Comitê de Redes Neurais Artificiais, o qual, hoje é tendência no mercado, que é uma área  altamente escalável que pode aprender diretamente a partir de dados brutos, usando camadas hierárquicas de redes, em combinação com técnicas de otimização baseadas em retro-propagação e gradiente descendente~\cite{barca2005treinamento}. 

A \textit{Machine Learning}, especialmente com o auxílio da computação GPU, que é encontrada na maioria dos hardwares pessoais, alcançou recentemente um imenso sucesso em aplicações comerciais e em Inteligência Artificial \cite{esteva2017dermatologist, moravvcik2017deepstack, van2016wavenet, 10.1007/978-3-319-44188-7_16}, e também tem sido aplicado em astrofísica \cite{shen2017denoising,PhysRevLett.120.141103,krastev2019real,gebhard2019convolutional,mukund2017transient,kim2015application,george2018deep,george2017glitch, george2017deepA}.

É neste contexto que se insere este trabalho, o qual, busca uma solução confiável simples e eficiente na busca por GWs de CBCs nos dados do aLIGO, utilizando Rede Neurais Artificiais.

\section{Organização do trabalho}
\label{sec:organizacao-do-trabalho}

O presente trabalho está organizado em 5 capítulos e um Apêndice, a saber: Introdução, Fundamentação Teórica, Metodologia, Resultados, Conclusão e Apêndice A. 

A seguir, são descritos resumidamente o conteúdo de cada um dos capítulos seguintes.

Na Fundamentação Teórica são apresentados conceitos sobre ondas gravitacionais e redes neurais artificiais. Na seção sobre ondas gravitacionais, é descrito uma breve explicação do que são ondas gravitacionais, suas fontes, tipos de sinais e como gerar formas de ondas com cálculos físicos. Na seção das redes neurais, descreve-se o neurônio artificial, algumas arquiteturas de redes e o processo de aprendizagem utilizado no trabalho (retro-propagação). 

No capítulo Metodologia, são descritas as ferramentas computacionais utilizadas, o tratamento dos dados obtidos do LIGO e o processo de geração de formas de ondas, juntamente com o processo de ajustes de parâmetros e de aprendizagem das redes neurais geradas (MLPs), e por ultimo, o arranjo das MLPs com o comitê de redes neurais.

O capítulo de Resultados apresenta os resultados obtidos com a aplicação dos dados reais do LIGO sobre o comitê de redes neurais treinadas, descreve as arquiteturas de redes neurais desenvolvidas e a que melhor se adequou ao problema de estudo, qual o melhor processo para a realização do treinamento das redes neurais e os critérios que levaram a esta descoberta.

No capítulo Conclusão são apresentados discussões sobre o trabalho como todo, ressaltando alguns detalhes observados nos Resultados, apresenta quais os ganhos e possíveis trabalhos futuros com descobertas já preestabelecidas.

É mostrado no Apêndice A os gráficos de score gerados pelo comitê de redes neurais para cada uma das ondas gravitacionais obtidas do aLIGO utilizadas neste trabalho.