A astronomia de ondas gravitacionais colocou em movimento uma revolução científica. Para aumentar ainda mais o alcance científico desse campo emergente, há uma necessidade premente de aumentar a profundidade e diminuir a complexidade computacional dos algoritmos de ondas gravitacionais. Técnicas de aprendizado de máquina, como redes neurais artificiais, já estão sendo usadas na astrofísica das ondas gravitacionais para detecção e caracterização de sinais de ondas gravitacionais. Para contribuir com esse esforço, este estudo mostra uma nova metodologia para detectar ondas gravitacionais, por Análise de Score. Foi desenvolvido um comitê de redes neurais para detectar ondas gravitacionais em tempo real. Utilizando os Scores criadas por um comitê de redes neurais aplicadas aos dados dos laboratório de Hanford e do laboratório de Livingston, para classificar um sinal com ondas gravitacionais e ruído. Produzindo Scores ao longo do tempo e observando novas informações nesse sinal, um platô, na fase de reverberação da onda gravitacional.

% foi criado um novo método altamente escalonável e computacionalmente leve para processamento de sinais de séries temporais, baseado em um sistema de comitê de redes neurais, para classificar rapidamente sinais de ondas gravitacionais de fusão de buracos negros em fluxos de dados de séries temporais altamente ruidosos dos dados típicos dos detectores de ondas gravitacionais do aLIGO. Os resultados demonstram o potencial de redes neurais artificiais para a detecção em tempo real de sinais de ondas gravitacionais, o que é essencial para um acompanhamento imediato para facilitar as pesquisas em tempo real de fontes de ondas gravitacionais e suas contrapartes eletromagnéticas que acompanham esses importantes eventos.

\palavraschave{Ondas Gravitacionais, Aprendizado de Máquina, Redes Neurais Artificiais, Buracos Negros, Análise de Score.}